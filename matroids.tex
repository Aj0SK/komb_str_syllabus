\chapter{Matroidy}

\section{Definícia, základné pojmy}

\begin{definition}

Dvojica $(X, \mathcal{N})$, kde $\mathcal{N} \in \powerset(X)$ a $\mathcal{N}$ je konečná, je matroid, ak sú splnené nasledujúce podmienky:
\begin{enumerate}
    \item $\varnothing \in \mathcal{N}$
    \item $N \in \mathcal{N} \wedge N' \subseteq N \Longrightarrow N' \in \mathcal{N}$
    \item $N, N' \in \mathcal{N} \wedge \ssize{N} < \ssize{N'} \Longrightarrow \exists x \in N' - N: N \cup \set{x} \in \mathcal{N}$
\end{enumerate}

Množiny z $\mathcal{N}$ voláme nezávislé množiny. Množiny mimo $\mathcal{N}$ voláme závislé.
\end{definition}

\begin{theorem}{(Matroid z vektorového priestoru)}\\
Nech $V_n(F) \cong F^n$ je vektorový priestor dimenzie $n < \infty$ nad (nie nutné konečným) poľom $F$. 
Nech $(\vec{x}_1, \ldots, \vec{x}_r)$ je postupnosť (nie nutné rôznych) vektorov z $V_n(F)$.
Nech $X := \set{1, \ldots, r}$, $\mathcal{N} := \set{Q ~|~ Q \subseteq X \wedge \set{\vec{x}_i ~|~ i \in Q} \text{ je nezávislá v } V_n(F) }$.
Potom dvojica $(X, \mathcal{N})$ je matroid.
\end{theorem}

\begin{theorem}{(Matroid z grafu)}\\
Nech $G = (V, E)$ je jednoduchý graf. Nech $X := E$. Nech množina hrán $A \subseteq E$ patrí do množiny $\mathcal{N}$ práve vtedy, keď
indukovaný graf neobsahuje kružnice. Potom dvojica $M(G) = (X, \mathcal{N})$ je matroid.
\end{theorem}

\TODO konstrukcia matroidu cez signovane grafy?

\begin{definition}
Nech $M = (X, \mathcal{N})$ je matroid a nech $A \subseteq X$. Množinu $B \subseteq A$ voláme bázou množiny $A$ v matroide $M$, ak 
je to maximálna (na inklúziu) nezávislá množina v $A$. Formálne, $B$ je bázou $A$ v matroide $M=(X, \mathcal{N})$, ak:
$$B \subseteq A \wedge B \in \mathcal{N} \wedge \left( \forall B' \supset B: B' \subseteq A \Longrightarrow B' \not\in\mathcal{N} \right)$$

Špeciálne, bázy množiny $X$ voláme bázy matroidu. Množinu báz matroidu $M$ znáčime ako $\mathcal{B}$.
\end{definition}

\begin{theorem}
Nech $(X, \mathcal{N})$ je matroid a $A \subseteq X$. Nech $N, N'$ sú bázy množiny $A$. Potom $\ssize{N} = \ssize{N'}$.
\end{theorem}

\begin{definition}
Nech $(X, \mathcal{N})$ je matroid. Hodnosťou množiny $A \subseteq X$ voláme veľkosť nejakej bázy $B$ množiny $A$ a znáčime ako $r(A) := \ssize{B}$.
\end{definition}

\begin{theorem}
Nech $(X, \mathcal{N})$ je matroid a $r:\powerset(X) \to \mathbb{N}_0$ je jeho hodnostná funkcia. Potom platí:
\begin{enumerate}
    \item $r(\varnothing) = 0$
    \item $r(\set{x}) \leq 1$
    \item $A \subseteq B \Longrightarrow r(A) \leq r(B)$
    \item $r(A \cup B) \leq r(A) + r(B) - r(A \cap B)$ (semimodularita)
\end{enumerate}

Navyše, ak nejaká funkcia $r':\powerset(X)\to\mathbb{N}_0$ spĺňa vyššie uvedené podmienky, tak existuje jediný matroid, ktorého
hodnostnou funkciou je práve $r'$.

\end{theorem}

\begin{theorem}
Nech $(X, \mathcal{N})$ je matroid a $\mathcal{B}$ je množina jeho báz. Potom platí:
\begin{enumerate}
    \item[B1:] žiadné 2 prvky množiny $\mathcal{B}$ nie sú v inklúzii
    \item[B2:] $B, B' \in \mathcal{B} \Longrightarrow \forall x \in B - B' ~\exists y \in B' - B: (B-\set{x}) \cup \set{y} \in \mathcal{B}$
\end{enumerate}

Navyše, ak množina $\mathcal{B}'$ spĺňa vyššie uvedené podmienky B1 a B2, tak existuje jediný matroid, ktorého množinou báz je práve $\mathcal{B}'$.
\end{theorem}

\begin{definition}
Nech $M = (X, \mathcal{N})$ je matroid a $A \subseteq X$. Prvok $x \in X$ voláme závislý od množiny $A$ v matroide $M$, ak $r(A) = r(A \cup \set{x})$.
\end{definition}

\begin{definition}
Nech $M = (X, \mathcal{N})$ je matroid a $A \subseteq X$. Úzaverom množiny $A$ v matroide $M$ voláme množinu $\bar{A}$ všetkých závislých prvkov od $A$. Formálne,
$$\bar{A} := \set{x \in X ~|~ r(A) = r(A \cup \set{x})}$$
\end{definition}

\begin{theorem}
Nech $M = (X, \mathcal{N})$ je matroid a $A \subseteq X$. Potom platí:
\begin{enumerate}
    \item $A \subseteq \bar{A}$
    \item $\bar{A} = \bar{\bar{A}}$
    \item $\bar{A} = \bigcup_{B \in \powerset(X)} \set{B ~|~ B \supseteq A \wedge r(B) = r(A)}$
    \item $r(A) = r(\bar{A})$
\end{enumerate}
\end{theorem}

\begin{theorem}
Nech $M = (X, \mathcal{N})$ je matroid a $\Phi:\powerset(X) \to \powerset(X)$ je jeho úzaverová funkcia (t.j. $\Phi(A) = \bar{A}$).
Potom platí:
\begin{enumerate}
    \item[U1:] $\forall A \subseteq X: A \in \bar{A}$
    \item[U2:] $A \subseteq \bar{B} \Longrightarrow \bar{A} \subseteq \bar{B}$
    \item[U3:] $x \not\in A \wedge x \in \overline{A \cup \set{y}} \Longrightarrow y \in \overline{A \cup \set{x}}$
\end{enumerate}

Navyše, ak existuje funkcia $\Phi': \powerset(A) \to \powerset(A)$, ktorá spĺňa podmienky U1 --- U3, tak existuje jediný matroid,
ktorého úzaverovou funkciou je práve $\Phi'$.
\end{theorem}

\begin{definition}
Nech $(X, \mathcal{N})$ je matroid. Množina $K \subseteq X$ sa volá kružnica, ak je to najmenšia (na inklúziu) závislá množina. Formálne,
množina $K \subseteq X$ je kružnica, ak:
$$ K \not\in\mathcal{N} \wedge \left(\forall K' \subseteq K: K' \in\mathcal{N} \right)$$ 
\noindent
Množinu všetkých kružníc matroidu označujeme ako $\mathcal{K}$.
\end{definition}

\begin{theorem}
Nech $(X, \mathcal{N})$ je matroid a $\mathcal{K}$ je množina všetkých jeho kružníc. Potom platí:
\begin{enumerate}
    \item[K1:] žiadne dva prvky množiny $\mathcal{K}$ nie sú v inklúzii
    \item[K2:] $K, K' \in \mathcal{K} \wedge K \neq K' \wedge \exists x \in K \cap K' \Longrightarrow \exists L \in \mathcal{K}: L \subseteq (K \cup K') - \set{x}$
\end{enumerate}

Navyše, ak existuje množina $\mathcal{K}'$, ktorá spĺňa podmienky K1 a K2, tak existuje jediný matroid, ktorého množinou kružníc je práve $\mathcal{K}'$.
\end{theorem}

\section{Dualita matroidov a triedy matroidov}

\begin{theorem}{(veta o dualite)}\\
Nech $M = (X, \mathcal{N})$ je matroid. Nech $\mathcal{B}$ je množina báz matroidu $M$ a $r:\powerset(X) \to \mathbb{N}_0$ je hodnostná funkcia matroidu $M$.
Ďalej nech:
\begin{itemize}
    \item $\mathcal{B}^* := \set{X - B ~|~ B \in \mathcal{B}}$
    \item $\mathcal{N}^* := \set{X - A ~|~ A \subseteq X \wedge r(A) = r(X)}$
    \item $r^*: \powerset(A) \to \mathbb{N}_0$ taká, že $r^*(A) := \ssize{A} - r(X) + r(X - A) = \ssize{A} - (r(X) - r(X-A))$
\end{itemize}

\noindent
Potom platí:
\begin{enumerate}
    \item množina $\mathcal{B}^*$ je systémom báz nejakého matroidu
    \item množina $\mathcal{N}^*$ je systémom nezávislých množín nejakého matroidu
    \item funkcia $r^*$ je hodnostnou funkciou nejakého matroidu
    \item navyše, všetky 3 vyššie uvedené matroidy sú rovnaké
\end{enumerate}

\noindent
Takýmto spôsobom zostrojený matroid sa volá duálny a značí sa ako $M^* = (X, \mathcal{N}^*)$.

\end{theorem}

\begin{definition}
Nech $X:=\set{1, \ldots, n}$, $\mathcal{N}:=\set{A ~|~ A \subseteq X \wedge \ssize{A} \leq k}$. Potom 
dvojica $U_k^n = (X, \mathcal{N})$ je matroid.
\end{definition}

\begin{theorem}
$(U_k^n)^* = U_{n-k}^n$
\end{theorem}

\begin{theorem}
Nech $M(G)$ je grafový matroidu grafu $G$. Potom nasledujúce podmienky sú ekvivalentné:
\begin{enumerate}
    \item $M^*$ je grafový matroid
    \item $G$ je planárny graf
\end{enumerate}
\end{theorem}

\begin{definition}
Nech $M=(X, \mathcal{N})$ je matroid a $F$ je pole. Matroid $M$ je $F$-reprezentovateľný, ak
existuje vektorový priestor $V$ konečnej dimenzie nad $F$ a zobrazenie $f:X \to V$ také, že
$$\forall A \in X: \left(A \in \mathcal{N} \Longleftrightarrow f(A) \text{ je lineárne nezávislá vo } V \right)$$ 
\end{definition}

\begin{definition}
Matroid je reprezentovateľný, ak je $F$-reprezentovateľný nad nejakým poľom $F$.
\end{definition}

\begin{definition}
Matroid je binárny, ak je $GF(2)$-reprezentovateľný.
\end{definition}

\begin{definition}
Matroid je regulárny, ak je $F$-reprezentovateľný nad každým poľom~$F$.
\end{definition}

\begin{theorem_hard}
Každý grafový matroid je regulárny.
\end{theorem_hard}

\TODO Cayleho grafy?

\TODO rezy a kruznice su kolme v kazdej reprezentacii?

\begin{definition}{(Zúženie matroidu)}\\
Nech $M = (X, \mathcal{N})$ je matroid a $Y \subseteq X$. 
Nech $\mathcal{N}_Y := \set{A ~|~ A \subseteq Y \wedge \exists B \in \mathcal{N}: A = B \cap Y}$.
Potom $\faktor{M}{Y} := (Y, \mathcal{N}_Y)$ je matroid a nazýva sa zúžením matroidu $M$ na množinu $Y$.
\end{definition}

\begin{definition}{(Kontrakcia matroidu)}\\
Nech $M = (X, \mathcal{N})$ je matroid a $Y \subseteq X$. 
Nech $A \subseteq Y$ patrí do systému $\mathcal{N}_Y$ práve vtedy, keď existuje 
báza $B$ množiny $X-Y$ v matroide $M$ taká, že $A \cup B \in \mathcal{N}$. Potom dvojica
$M.Y := (Y, \mathcal{N}_Y)$ je matroid a nazýva sa kontrakciou matroidu $M$ na množinu $Y$.
\end{definition}

\begin{theorem}
$(M.Y)^* = \faktor{M^*}{Y}$
\end{theorem}

\begin{definition}
Matroid $M'$ je minorom matroidu $M$, ak sa matroid $M'$ dá dostať z matroidu $M$ pomocou postupnosti zúžení a kontrakcií.
\end{definition}

\TODO Fannov matroid

\begin{theorem_hard}{(Charakterizácia tried matroidov)}\\

Oznáčme Fannov matroid ako $\mathcal{F}$.

\begin{enumerate}
    \item matroid M je binárny $\Longleftrightarrow$ $U_2^4$ nie je minorom matroidu~M.
    \item matroid M je regulárny $\Longleftrightarrow$ $U_2^4$, $\mathcal{F}$, $\mathcal{F}^*$ nie sú minormi matroidu~M.
    \item matroid M je grafový $\Longleftrightarrow$ $U_2^4$, $\mathcal{F}$, $\mathcal{F}^*$, $M^*(K_{3,3})$, $M^*(K_{5})$ nie sú minormi matroidu~M.
    \item matroid M je kografový $\Longleftrightarrow$ $U_2^4$, $\mathcal{F}$, $\mathcal{F}^*$, $M(K_{3,3})$, $M(K_{5})$ nie sú minormi matroidu~M.
    \item matroid M je planárny $\Longleftrightarrow$ matroid M je grafový a kografový.
\end{enumerate}
\end{theorem_hard}


\section{Matroidové algoritmy}

\begin{definition}

Problém maximálnej množiny je trojica $(X, \mathcal{M}, c)$, kde 
$X = {x_1, \ldots, x_n}$ je množina objektov, 
$\mathcal{M} \subseteq \powerset(X)$ je množina prípustných riešení a $c: X \to \mathbb{R}^+ \cup \set{0}$
je cenová funkcia, rozširiteľná na $\powerset(X)$, a to takým spôsobom: $\forall A \in \powerset(X): c(A) := \sum_{x_i \in A} c(x_i) $.
Riešením problému maximálnej množiny je množina $M^* \in \mathcal{M}$ s maximálnou cenou. Formálne,
$$M^* := \underset{M \in \mathcal{M}}{\arg\max}~c(M)$$

\end{definition}

\begin{definition}{(Pažravý algoritmus)}\\
Nech $(X, \mathcal{M}, c)$ je problém maximálnej množiny. Potom nasledovný algoritmus je pažravým algoritmom pre najdenie riešenia
daného problému:
\begin{enumerate}
    \item $M_0 := \varnothing$
    \item $M_{i+1} := M_i \cup \set{x}$, ak $x$ spĺňa nasledovné podmienky:
    \begin{enumerate}
        \item $x \not\in M_i$
        \item $M_i \cup \set{x} \subseteq M' \in \mathcal{M}$ (t.j. existuje také $M' \in \mathcal{M}$)
        \item $x$ má spomedzi všetkých prvkov, ktoré splňajú predchádzajúce podmienky, maximálnu cenu $c(x)$
    \end{enumerate}
    \item Opakujeme krok 2. Ak $x$, vyhovujúce všetkým podmienkam z druhého kroku, neexistuje, tak algoritmus končí a odpoveďou je posledná množina $M_i$.
\end{enumerate}
\end{definition}

\begin{theorem}{(Vzťah matroidov a pažravých algoritmov)}\\
Nech $X$ je konečná množina, $\mathcal{M} \subseteq \powerset(X)$. Potom nasledujúce podmienky sú ekvivalentné:
\begin{enumerate}
    \item pre každé nezáporné ohodnotenie $c$ množiny $X$ pažravý algoritmus nájde optimálne riešenie
    \item existuje matroid na množine $X$ taký, že $\mathcal{M}$ je systémom báz daného matroidu 
\end{enumerate}
\end{theorem}
