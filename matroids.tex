\chapter{Matroidy}

\section{Definícia, základné pojmy}

\begin{definition}

Dvojica $(X, \mathcal{N})$, kde $\mathcal{N} \in \powerset(X)$ a $\mathcal{N}$ je konečná, je matroid, ak sú splnené nasledujúce podmienky:
\begin{enumerate}
    \item $\varnothing \in \mathcal{N}$
    \item $N \in \mathcal{N} \wedge N' \subseteq N \Longrightarrow N' \in \mathcal{N}$
    \item $N, N' \in \mathcal{N} \wedge \ssize{N} < \ssize{N'} \Longrightarrow \exists x \in N' - N: N \cup \set{x} \in \mathcal{N}$
\end{enumerate}

Množiny z $\mathcal{N}$ voláme nezávislé množiny. 
\end{definition}

\begin{definition}
Nech $M = (X, \mathcal{N})$ je matroid a nech $A \subseteq X$. Množinu $B \subseteq A$ voláme bázou množiny $A$ v matroide $M$, ak 
je to maximálna (na inklúziu) nezávislá množina v $A$. Formálne, $B$ je bázou $A$ v matroide $M=(X, \mathcal{N})$, ak:
$$B \subseteq A \wedge B \in \mathcal{N} \wedge \left( \forall B' \supset B: B' \subseteq A \Longrightarrow B' \not\in\mathcal{N} \right)$$

Špeciálne, bázy množiny $X$ voláme bázy matroidu. Množinu báz matroidu $M$ znáčime ako $\mathcal{B}$.
\end{definition}

\begin{theorem}
Nech $(X, \mathcal{N})$ je matroid a $A \subseteq X$. Nech $N, N'$ sú bázy množiny $A$. Potom $\ssize{N} = \ssize{N'}$.
\end{theorem}

\begin{definition}
Nech $(X, \mathcal{N})$ je matroid. Hodnosťou množiny $A \subseteq X$ voláme veľkosť nejakej bázy $B$ množiny $A$ a znáčime ako $r(A) := \ssize{B}$.
\end{definition}

\begin{theorem}
Nech $(X, \mathcal{N})$ je matroid a $r:\powerset(X) \to \mathbb{N}_0$ je jeho hodnostná funkcia. Potom platí:
\begin{enumerate}
    \item $r(\varnothing) = 0$
    \item $r(\set{x}) \leq 1$
    \item $A \subseteq B \Longrightarrow r(A) \leq r(B)$
    \item $r(A \cup B) \leq r(A) + r(B) - r(A \cap B)$ (semimodularita)
\end{enumerate}

Navyše, ak nejaká funkcia $r':\powerset(X)\to\mathbb{N}_0$ spĺňa vyššie uvedené podmienky, tak existuje jediný matroid, ktorého
hodnostnou funkciou je práve $r'$.

\end{theorem}

\begin{theorem}
Nech $(X, \mathcal{N})$ je matroid a $\mathcal{B}$ je množina jeho báz. Potom platí:
\begin{enumerate}
    \item[B1:] žiadné 2 prvky množiny $\mathcal{B}$ nie sú v inklúzii
    \item[B2:] $B, B' \in \mathcal{B} \Longrightarrow \forall x \in B - B' ~\exists y \in B' - B: (B-\set{x}) \cup \set{y} \in \mathcal{B}$
\end{enumerate}

Navyše, ak množina $\mathcal{B}'$ spĺňa vyššie uvedené podmienky B1 a B2, tak existuje jediný matroid, ktorého množinou báz je práve $\mathcal{B}'$.
\end{theorem}

\TODO uzaver

\TODO kruznice

\TODO priklady matroidov: cez vektory, cez grafy, cez signovane grafy

\section{Dualita matroidov}

\TODO veta o dualite

\TODO dualne grafy a triedy matroidov (reprezentovatelne, regularne, grafove, etc.)

\TODO zuzenie matroidu, kontrakcia matroidu, minor matroidu

\TODO charakteristiky tried matroidov a vztah s planaritou grafov



\section{Matroidové algoritmy}

\TODO definicia pazraveho algoritmu

\TODO veta o optimalite pazraveho algoritmu
