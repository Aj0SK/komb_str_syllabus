\chapter{Latinské štvorce}

\section{Definícia, základné vlastnosti}

\begin{definition}
Tabuľka rozmerov $n\times n$ s prvkami z $\set{1, \ldots, n}$ je latinský~štvorec rádu $n$, ak platí:
\begin{enumerate}
    \item v každom riadku sa vyskytuje všetkých $n$ rôznych symbolov
    \item v každom stĺpci sa vyskytuje všetkých $n$ rôznych symbolov
\end{enumerate}
\end{definition}

Symbolom $S_n$ značíme grupu permutácií veľkosti $n$.

\begin{exercise}
Zostrojte latinský štvorec rádu 5.
\end{exercise}

\begin{exercise}
Zostrojte latinský štvorec rádu 7.
\end{exercise}

\begin{exercise}
Napíšte program na generovanie latinských štvorcov, ktorý bude generovať štvorec po riadkoch preberaním všetkých možností.
\end{exercise}

\begin{definition}
\label{def:permdist}
Nech $\phi, \psi \in S_n$ sú permutácie veľkosti $n$. Potom vzdialenosť $dist(\phi, \psi)$ dvoch permutácií definujeme ako počet
prvkov, ktoré dané permutácie zobrazia rôzne. Formálne, $$dist(\phi, \psi) := \left| \set{x ~|~ x \in \set{1, \ldots, n} \wedge \phi(x) \neq \psi(x) } \right| $$
\end{definition}

\begin{definition}
Nech $\phi \in S_n$ je permutácia veľkosti $n$. Potom $Fix(\phi)$ je množina všetkých pevných bodov permutácie $\phi$. Formálne,
$$Fix(\phi) := \set{x ~|~ x \in \set{1, \ldots, n} \wedge \phi(x) = x}$$ 
\end{definition}

\begin{theorem}
\label{thm:permdist}
Nech $\phi, \psi \in S_n$ sú permutácie veľkosti $n$. Potom platí:

\begin{enumerate}
    \item $\forall \lambda \in S_n: dist(\lambda \phi, \lambda \psi) = dist(\phi, \psi)$ 
    \item $dist(\phi, \psi) = dist(1, \phi^{-1} \psi) = n - |Fix(\phi^{-1} \psi)|$
\end{enumerate}
\emph{(Poznámka: jednotkou v tomto kontexte označujeme neutrálny prvok grupy $S_n$, t.j. identickú permutáciu)}
\end{theorem}
\begin{proof}
Začneme dôkazom prvého tvrdenia. Chceme ukázať, že vzdialenosti dvoch párov permutácií sú rovnaké. 
To znamená, že počty prvkov, v ktorých sa ich hodnoty líšia, je rovnaký.
Prvú množinu prvkov označíme ako $A$, druhú ako $B$. 

Formálne, nech $$A := \set{x | x \in \set{1, \ldots, n} \wedge \phi(x) \neq \psi(x)}$$ (t.j. $dist(\phi, \psi) = |A|$ z definície vzdialenosti \ref{def:permdist}) a $$B := \set{y | y \in \set{1, \ldots, n} \wedge \phi(\lambda(y)) \neq \psi(\lambda(y))}$$ (t.j. $dist(\lambda\phi, \lambda\psi) = |B|$ z definície vzdialenosti \ref{def:permdist}).

Najprv ukážeme, že platí $|A| \geq |B|$, následne $|B| \geq |A|$.
Z toho už platnosť prvého tvrdenia z vety bude očividná.

Nech $x \in A$. 
Z definície množiny $A$ vyplýva, že $\phi(x) \neq \psi(x)$. 
Nech $y := \lambda^{-1}(x)$. Potom platí, že $\phi(\lambda(y)) \neq \psi(\lambda(y))$, čiže $y \in B$ z definície množiny $B$.
Keďže $\lambda^{-1}$ je permutácia, tak je injektívnym zobrazením. 
To znamená, že rôzne hodnoty $x$ premietne do rôznych hodnôt $y$. 
Z toho vyplýva, že veľkosť množiny $B$ je aspoň tak veľká, ako množiny $A$, t.j. $|A| \leq |B|$.

Ukážeme teraz druhú nerovnosť. Nech $y \in B$. Potom (z definície množiny $B$) platí $\phi(\lambda(y)) \neq \psi(\lambda(y))$. 
Teda, z definície množiny $A$, $\lambda(y) \in A$.
Analogicky (z injektivity permutácie $\lambda$) vyplýva, že $|B| \leq |A|$.
Týmto je dôkaz prvého tvrdenia z~dokazovanej vety ukončený.

Dôkaz druhého tvrdenia z vety prenechávame čitateľovi ako samostatné cvičenie.
\end{proof}

\begin{exercise}
Dokážte tvrdenie 2 z vety \ref{thm:permdist} (\emph{hint: použite prvé tvrdenie z tejto vety}).
\end{exercise}

\begin{exercise}
Dokážte, že $\forall \phi, \psi, \lambda \in S_n: dist(\phi \lambda, \psi \lambda) = dist(\phi, \psi)$ (\emph{hint: využite bijektivnosť permutácie $\lambda$ a techniku z dôkazu vety \ref{thm:permdist}}).
\end{exercise}

\begin{theorem}
    Funkcia $dist(\phi, \psi)$ je metrikou priestoru $S_n$, t.j. ona spĺňa nasledujúce podmienky:
    \begin{enumerate}
        \item $ \forall \phi, \psi \in S_n: dist(\phi, \psi) = 0 \Leftrightarrow \phi = \psi$
        \item $ \forall \phi, \psi \in S_n: dist(\phi, \psi) = dist(\psi, \phi)$ (symetria)
        \item $ \forall \phi, \psi, \lambda \in S_n: dist(\phi, \psi) + dist(\psi, \lambda) \geq dist(\phi, \lambda)$ (trojuholníková nerovnosť)  
    \end{enumerate}
\end{theorem}

\begin{proof}
Prvé dve tvrdenia sú očividné a nepotrebujú špeciálny dôkaz.
Pozrime sa teda na tretie tvrdenie o trojuholníkovej nerovnosti.

Nech $\phi_1, \phi_2, \phi_3 \in S_n$. 
Treba dokázať, že $$dist(\phi_1, \phi_2) + dist(\phi_2, \phi_3) \overset{?}{\geq} dist(\phi_1, \phi_3).$$
Z vety \ref{thm:permdist} táto nerovnosť je ekvivalentná s nerovnosťou $$(n - |Fix(\phi_1^{-1}\phi_2)|) + (n - |Fix(\phi_2^{-1}\phi_3)|) \overset{?}{\geq} n - |Fix(\phi_1^{-1}\phi_3)|.$$
Ak komplement množiny $A$ (vzhľadom na základnú množinu $\set{1, \ldots, n}$) označíme ako $\overline{A}$, tak túto nerovnosť môžeme prepísať do ďalšieho ekvivalentného tvaru: $$|\overline{Fix(\phi_1^{-1}\phi_2)}| + |\overline{Fix(\phi_2^{-1}\phi_3)}| \overset{?}{\geq} |\overline{Fix(\phi_1^{-1}\phi_3)}|.$$

Dokážeme si pomocné tvrdenie\footnote{niektorí ľudia v týchto okamihoch zvyknú nahlas povedať "Heuréka!"}: $$\overline{Fix(\phi_1^{-1}\phi_2)} \cup \overline{Fix(\phi_2^{-1}\phi_3)} \overset{?}{\supseteq} \overline{Fix(\phi_1^{-1}\phi_3)}.$$

Všimnite si, že ak daná inklúzia platí, tak z nej vyplývajú aj predchádzajúce tvrdenia (kvôli $A \cup B \supseteq C \Longrightarrow |A \cup B| \geq |C| \Longrightarrow |A| + |B| \geq |C|$).
Môžeme toto tvrdenie prepísať do ekvivalentného tvaru, odstrániac znaky komplementu:
$$Fix(\phi_1^{-1}\phi_2) \cap Fix(\phi_2^{-1}\phi_3) \overset{?}{\subseteq} Fix(\phi_1^{-1}\phi_3).$$

Túto inklúziu dokážeme priamo. 
Nech $x \in Fix(\phi_1^{-1}\phi_2) \cap Fix(\phi_2^{-1}\phi_3)$. 
Potom $x = \phi_2(\phi_1^{-1}(x))$ a $x = \phi_3(\phi_2^{-1}(x))$.
Dosadením prvej rovnosti do druhej dostaneme $x = \phi_3(\phi_2^{-1}(\phi_2(\phi_1^{-1}(x)))) = \phi_3(\phi_1^{-1}(x))$, čiže $x$ je aj pevným bodom permutácie $\phi_1^{-1} \phi_3$, t.j. $x \in Fix(\phi_1^{-1}\phi_3)$.
Týmto je dôkaz pomocného tvrdenia, a s tým aj celej vety, ukončený.
\end{proof}

\begin{definition}
Latinský obdĺžnik rozmerov $k \times n$ je postupnosť $L = [\phi_1, \phi_2, \ldots, \phi_k]$ permutácií z $S_n$ takých,
že všetky sú vo vzdialenosti $n$. Formálne, 
$$\forall i, j \in \set{1, \ldots, n}: i \neq j \Rightarrow dist(\phi_i, \phi_j) = n$$
\end{definition}

\begin{definition}{(Iná definícia latinských štvorcov)}
Latinský štvorec rádu $n$ je latinský obdĺžnik typu $k \times n$ s maximálnou dĺžkou postupnosti. Inak povedané,
latinský štvorec rádu $n$ je postupnosť $n$ permutácií z $S_n$, ktoré sú od seba vzdialené $n$.
\end{definition}

\begin{definition}
    Nech X je množina a $\mathcal{X} = \set{X_1, \ldots, X_k}, \forall i \in \set{1, \ldots, k}: X_i \subseteq X$ je systém jej podmnožín.
    Systém rozličných reprezentantov pre $\mathcal{X}$ je postupnosť $x_1, \ldots x_k, \forall i \in \set{1, \ldots, k}: x_i \in X_i$ 
    a zároveň sú všetky jej prvky rôzne, teda $\forall i, j \in \set{1, \ldots, k}: i \neq j \Rightarrow  x_i \neq x_j$.
\end{definition}

\begin{theorem}{(Hallova veta pre množiny)}
    Nech $\mathcal{X}$ je systém podmnožín množiny $X$. 
    Ak $\forall \mathcal{Y} = \set{Y_1, \ldots, Y_m} \subseteq \mathcal{X}: |\bigcup_{Y \in \mathcal{Y}} Y| > m$ tak pre 
    $\mathcal{X}$ existuje systém rozličných reprezentantov.
\end{theorem}
\begin{proof}
Cez Hallovu vetu pre bipartitné grafy. 
Vytvoríme bipartitný graf, ktorého partícia $A$ obsahuje jeden vrchol pre každú množinu 
z $\mathcal{X}$ a partícia $B$ obsahuje jeden vrchol pre každý prvok z $X$. Každý vrchol množiny spojíme s vrcholmi prvkov, 
ktoré obsahuje, teda $(X_i, x_j) \in E(G) \Leftrightarrow x_j \in X_i$. 
Z Hallovej vety o párení potom existuje párenie pokrývajúce všetky vrcholy $A$,
ak z každej podmnožiny vrcholov $A$ veľkosti $m$ vychádza aspoň $m$ hrán.  
\end{proof}

\begin{theorem}
    Každý latinský obdĺžnik sa dá doplniť na latinský štvorec.
\end{theorem}
\begin{proof}
    Pomocou Hallovej vety dokážeme, že do každého latinského obdĺžnika $k \times n, k < n$ vieme pridať ďalší riadok.
    Pre $i$-ty stĺpec obdĺžnika definujeme množinu kandidátov $X_i$ ako prvky, ktoré sa v stĺpci nenachádzajú.
    Z latinskej vlastnosti vyplýva, že do každého stĺpca sa dá doplniť práve $n-k$ prvkov, 
    teda $\forall i \in \set{1, \ldots, n}: |X_i| = n-k$. Opačne, každý prvok sa dá doplniť do $n-k$ stĺpcov. 
    Bipartitný graf zodpovedajúci $\mathcal{X} = \set{X_1, \ldots X_n}$ je teda $(n-k)$-regulárny. 
    Z každej množiny stĺpcov veľkosti $m$ tak vychádza $m(n-k)$ hrán, ktoré musia v druhej partícii vchádzať do $m$ vrcholov. 
    Keďže je splnená Hallova podmienka, existuje systém reprezentantov $\mathcal{X}$, ktorý vieme použiť ako $(k+1)$-vý riadok
    latinského obdĺžnika.
\end{proof}

\section{Ortogonálne latinské štvorce}

\begin{definition}
Nech $l = [\phi_1, \ldots, \phi_n]$ a $l' = [\psi_1, \ldots, \psi_n]$ sú latinské štvorce rádu $n$. Hovoríme, že
$l$ a $l'$ sú ortogonálne (znáčime ako $l \perp l'$), ak platí: 
$$\forall i,j,k,l \in \set{1, \ldots, n}: (i, j) \neq (k, l) \Longrightarrow (\phi_i(j), \psi_i(j)) \neq  (\phi_k(l), \psi_k(l))$$

\end{definition}


\begin{theorem}
Nech $l = [\phi_1, \ldots, \phi_n]$ a $l' = [\psi_1, \ldots, \psi_n]$ sú latinské štvorce rádu $n$.
Zavedieme nasledovné značenia:
\begin{itemize}
    \item Nech $\lambda \in S_n$, potom $\lambda l := [\lambda \phi_1, \ldots, \lambda \phi_n]$ ($\lambda l$ je tiež latinský štvorec). 
    \item Nech kompozícia $l$ a $l'$ je definovaná ako $l \circ l' := [\phi_1 \psi_1, \ldots, \phi_n \psi_n]$.
\end{itemize}

Potom platí:

\begin{enumerate}
    \item $l \perp l' \Leftrightarrow [\psi_1 \phi_1^{-1}, \ldots, \psi_n \phi_n^{-1}]$ je latinský štvorec
    \item Ak $\lambda, \rho \in S_n$ a $l \perp l'$,  tak aj $\lambda l \perp \rho l'$
    \item $l \perp l' \Leftrightarrow$ existuje latinský štvorec $l''$ taký, že $l' = l'' \circ l$ 
\end{enumerate}
    
\end{theorem}

\begin{definition}
Množina latinských štvorcov rádu $n$ $\set{l_1, \ldots, l_r}$ je maximálna, ak $\forall i \neq j: l_i \perp l_j$ a 
nedá sa doplniť ďalším latinským štvorcom bez porušenia prvej podmienky. 
\end{definition}

\begin{theorem}
Maximálna množina latinských štvorcov rádu $n$ má najviac $n-1$ prvkov.
\end{theorem}

\begin{definition}
Latinský štvorec je v normánom tvare, ak prvý riadok tabuľky je rovný $(1, \ldots, n)$ a prvý stlpec je rovný $(1, \ldots, n)^T$.
\end{definition}

\begin{definition}
Latinské štvorce $l$ a $l'$ sú izotopické, ak sa dajú permutáciou riadkov, stĺpcov a názvov prvkov previesť na rovnaký latinský štvorec v normálnom tvare.
\end{definition}

\begin{remark}
Latinský štvorec v normálnom tvare zodpovedá tabuľke binárnej operácie kvazigrupy 
(kvazigrupa je množina s invertovateľnou binárnou operáciou a neutrálnym prvkom\footnote{dá sa to neformálne predstaviť ako grupu bez zaručenej asociativity}).

Platí, že 2 kvazigrupy sú izomorfné práve vtedy, keď príslušné latinské štvorce sú izotopické.
\end{remark}

\begin{definition}
Latinský štvorec si vieme predstaviť ako maximálnu (na inklúziu) množinu $A$ trojíc $(r,c,s) \in \set{1, \ldots, n}^3$, 
kde $r$ zodpovedá číslu riadku, $c$ zodpovedá číslu stĺpca a $s$ zodpovedá hodnote v políčku $(i, j)$, 
takú, že platí:
\begin{itemize}
    \item všetky dvojice $(r, c)$ sú rôzne (''máme $n^2$ políčok'')
    \item všetky dvojice $(r, s)$ sú rôzne (''v každom riadku sa vyskytnú všetky hodnoty od $1$ po $n$'')
    \item všetky dvojice $(c, s)$ sú rôzne (''v každom stĺpci sa vyskytnú všetky hodnoty od $1$ po $n$'')
\end{itemize}  

Konjugáciou latinského štvorca voláme množinu trojíc $A'$, ktorá vznikne z $A$ permutáciou trojíc. Formálne,
nech $\lambda \in S_3$ je permutácia veľkosti $3$, potom $$A' = \set{ (a_{\lambda(1)}, a_{\lambda(2)}, a_{\lambda(3)} ) ~|~ (a_1, a_2, a_3) \in A}$$

Latinské štvorce, ktoré sa dajú jeden z druhého dostať pomocou konjugácie, voláme \emph{konjugované}.
Latinské štvorce, ktoré sa dajú jeden z druhého dostať pomocou konjugácie a izotopie, voláme \emph{paratopické}.

\end{definition}


\begin{theorem}{(Stevens, 1935)}
Ak $n = p^\alpha$, kde $p$ je prvočíslo, tak maximálna množina latinských štvorcov má $n-1$ prvkov.
\end{theorem}

\begin{construction}
Číslo $n$ je mocninou prvočísla, preto existuje konečné pole $F := GF(n)$ príslušnej veľkosti. 
Očíslujeme prvky poľa $F$ v ľubovoľnom poradí, ale nech $a_0 = 0$.

$k$-tý latinský štvorec si označme ako $l_k = \left(a_{ij}^{(k)}\right)$.

$a_{ij}^{(k)} := a_i a_k + a_j$

\end{construction}

\begin{definition}
$MOLS(n)$ je mohutnosť maximálnej množiny latinských štvorcov rádu $n$.
\end{definition}

\begin{remark}
$MOLS(6) = 1$
\end{remark}

\begin{theorem_hard}{(Bose, Parker, Schrickhande, 1960)}

$\forall n \geq 3 \wedge n \neq 6: MOLS(n) \geq 2$

\end{theorem_hard}

\begin{theorem}
$MOLS(n_1) \geq m \wedge MOLS(n_2) \geq m \Rightarrow MOLS(n_1 n_2) \geq m$
\end{theorem}

\begin{construction}
$k$-tý latinský štvorec rádu $n_1 n_2$ sa dá získať pomocou Kroneckerovho súčinu $k$-tých príslušných latinských štvorcov rádu $n_1$ a $n_2$.

Formálne, nech $l_1, \ldots, l_m$ sú ortogonálne latinské štvorce rádu $n_1$ a $l'_1, \ldots, l'_m$ sú ortogonálne latinské štvorce rádu $n_2$.
Potom množina matíc $\set{l_k \otimes l'_k ~|~ k \leq m}$, kde $\otimes$ je Kroneckerov súčin matíc, je množina ortogonálnych latinských štvorcov rádu $n_1 n_2$.
\end{construction}

\begin{corollary}
$$n = p_1^{\alpha_1} p_2^{\alpha_2} \ldots p_r^{\alpha_r} \Rightarrow MOLS(n) \geq \min_{i \leq r} (p_i^{\alpha_i} - 1)$$
\end{corollary}

\begin{theorem}
$$n = 2m - 1 \Rightarrow MOLS(n) \geq 2$$
\end{theorem}

\begin{construction}

Pohybujeme sa v cyklickej grupe $(\mathbb{Z}_n, +) = \set{0, \ldots, n-1}$.

\begin{align*}
A := (a_{ij}),~& a_{ij} := m (i+j)~(mod~n) \\
B := (b_{ij}),~& b_{ij} := (i-j)  ~(mod~n)    
\end{align*}
\end{construction}
