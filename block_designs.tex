\chapter{Blokové plány}

\section{Definícia, základné vlastnosti}

\begin{definition}

Vyvážený nekompletný blokový plán (angl. \emph{balanced incomplete block design}) $BIBD(v, b, r, k, \lambda)$ 
je usporiadaná dvojica $(X, \mathcal{B})$, kde $X$ je množina objektov a $\mathcal{B} \subset \powerset(X)$ je 
množina podmnožín objektov (tieto podmnožiny voláme \emph{bloky}), pričom sú splnené nasledujúce podmienky:

\begin{enumerate}
    \item $v = |X|$ je mohutnosť množiny objektov.
    \item $b = |\mathcal{B}|$ je mohutnosť množiny blokov.
    \item každý blok má mohutnosť $k$.
    \item každý bod je obsiahnutý v práve $r$ blokoch.
    \item každá dvojica bodov sa vyskytuje v práve $\lambda$ blokoch. 
\end{enumerate}
\end{definition}

\begin{theorem}
$\exists BIBD(v, b, r, k, \lambda) \Longleftrightarrow $ $\lambda$-násobný kompletný multigraf rádu~$v$ $\lambda K_v$
sa dá rozložiť na $b$ hranovo disjunktných klík rádu~$k$ ($K_k$).
\end{theorem}

\begin{theorem}

Nech existuje $BIBD(v, b, r, k, \lambda)$. Potom:
\begin{enumerate}
    \item $vr = bk$
    \item $\lambda (v-1) = r (k-1)$
\end{enumerate}
\end{theorem}

\begin{corollary}
Preto namiesto značenia $BIBD(v, b, r, k, \lambda)$ budeme použivať značenie $BIBD(v, k, \lambda)$ (zvyšné parametre vieme dorátať).
\end{corollary}

\begin{theorem}

Nech existuje $BIBD(v, b,r, k, \lambda)$, kde $X = \set{x_1, x_2, \ldots, x_v}$ a $\mathcal{B} = \set{B_1, \ldots, B_b}$. 
Nech matica incidencie $A \in \set{0, 1}^{v \times b}$ je matica typu $v\times b$, kde $A_{ij} = 1$ práve vtedy, keď $x_i \in B_j$.
Potom $A A^T = (r-\lambda) I_v + \lambda J_{v}$, 
kde $I_v$ je matica identity rádu $v$ a 
$J_v$ je matica jednotiek typu $v \times v$.
\end{theorem}

\begin{lemma}
Nech $A$ je matica incidencie blokového plánu $BIBD(v, b,r, k, \lambda)$. Potom $det(AA^T) = (r-\lambda)^{v-1} (v\lambda - \lambda + r)$.
\end{lemma}

\begin{corollary}
Ak $BIBD(v, b,r, k, \lambda)$ je blokový plán a $b=v$, tak matica incidencie $A$ je regulárna a matici $A^T$ tiež zodpovedá nejaký blokový plán.
\end{corollary}

\begin{remark}
Blokové plány také, že $b = v$, voláme symetrické.
\end{remark}


\begin{theorem}{(Fisherova nerovnosť)}
Nech existuje blokový plán $BIBD(v, b,r, k, \lambda)$. Potom $b \geq v$.
\end{theorem}

\section{Cyklické blokové plány a diferenčné množiny}
\section{Hadamardove matice}
\section{Konečné projektívne roviny}
\section{Steinerovské systémy trojíc, zovšeobecnenia}
