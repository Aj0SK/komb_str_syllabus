\chapter{Blokové plány}

\section{Definícia, základné vlastnosti}

\begin{definition}

Vyvážený nekompletný blokový plán (angl. \emph{balanced incomplete block design}) $BIBD(v, b, r, k, \lambda)$ 
je usporiadaná dvojica $(X, \mathcal{B})$, kde $X$ je množina objektov a $\mathcal{B} \subset \powerset(X)$ je 
množina podmnožín objektov (tieto podmnožiny voláme \emph{bloky}), pričom sú splnené nasledujúce podmienky:

\begin{enumerate}
    \item $v = |X|$ je mohutnosť množiny objektov.
    \item $b = |\mathcal{B}|$ je mohutnosť množiny blokov.
    \item každý blok má mohutnosť $k$.
    \item každý bod je obsiahnutý v práve $r$ blokoch.
    \item každá dvojica bodov sa vyskytuje v práve $\lambda$ blokoch. 
\end{enumerate}
\end{definition}

\begin{theorem}
$\exists BIBD(v, b, r, k, \lambda) \Longleftrightarrow $ $\lambda$-násobný kompletný multigraf rádu~$v$ $\lambda K_v$
sa dá rozložiť na $b$ hranovo disjunktných klík rádu~$k$ ($K_k$).
\end{theorem}

\begin{theorem}

Nech existuje $BIBD(v, b, r, k, \lambda)$. Potom:
\begin{enumerate}
    \item $vr = bk$
    \item $\lambda (v-1) = r (k-1)$
\end{enumerate}
\end{theorem}

\begin{corollary}
Preto namiesto značenia $BIBD(v, b, r, k, \lambda)$ budeme často
použivať značenie $BIBD(v, k, \lambda)$, nakoľko 
zvyšné parametre vieme dorátať: 
$$r := \dfrac{\lambda (v-1)}{k-1},~ b := \dfrac{\lambda v (v-1)}{k (k-1)}$$
\end{corollary}

\begin{theorem}

Nech existuje $BIBD(v, b,r, k, \lambda)$, kde $X = \set{x_1, x_2, \ldots, x_v}$ a $\mathcal{B} = \set{B_1, \ldots, B_b}$. 
Nech matica incidencie $A \in \set{0, 1}^{v \times b}$ je matica typu $v\times b$, kde $A_{ij} = 1$ práve vtedy, keď $x_i \in B_j$.
Potom $A A^T = (r-\lambda) I_v + \lambda J_{v}$, 
kde $I_v$ je matica identity rádu $v$ a 
$J_v$ je matica jednotiek typu $v \times v$.
\end{theorem}

\begin{lemma}
Nech $A$ je matica incidencie blokového plánu $BIBD(v, b,r, k, \lambda)$. Potom $det(AA^T) = (r-\lambda)^{v-1} (v\lambda - \lambda + r)$.
\end{lemma}

\begin{corollary}
Ak $BIBD(v, b,r, k, \lambda)$ je blokový plán a $b=v$, tak matica incidencie $A$ je regulárna a matici $A^T$ tiež zodpovedá nejaký blokový plán.
\end{corollary}

\begin{remark}
Blokové plány také, že $b = v$, voláme symetrické.
\end{remark}


\begin{theorem}{(Fisherova nerovnosť)}
Nech existuje blokový plán $BIBD(v, b,r, k, \lambda)$. Potom $b \geq v$.
\end{theorem}

\begin{corollary}
Nech existuje blokový plán $BIBD(v, b,r, k, \lambda)$. Potom $r \geq k$.
\end{corollary}

\section{Cyklické blokové plány a diferenčné množiny}

\begin{definition}

Množina $D = \set{d_1, \ldots, d_k} \subset \mathbb{Z}_v$ mohutnosti $k$ sa volá $(v, k, \lambda)$-diferenčnou množinou, ak 
pre každý nenulový prvok $a \in \mathbb{Z}_v$ existuje práve $\lambda$ usporiadaných dvojíc $(d_i, d_j) \in D^2$ takých, že
$d_i - d_j \equiv a \mod v$. 
\end{definition}

\begin{remark}

Množina $\set{0, 1, 3}$ je $(7, 3, 1)$-diferenčnou množinou.

\end{remark}

\begin{remark}

Podobným spôsobom je možné definovať diferenčné množiny nad konečnými grupami rádu $v$.

\end{remark}

\begin{definition}
$(v, k, \lambda)$-BIBD je cyklický, ak existuje permutácia s cyklom dlžky $v$ taká, že zachováva bloky\footnote{
bijektívne zobrazenia množiny na ňu samu, ktoré zachovávajú vzťahy medzi objektami, sa všeobecne nazývajú \emph{automorfizmy}}. 
Formálne, blokový plán je cyklický, ak 
existuje permutácia  $\phi \in S_v$ s cyklom dlžky $v$ taká, že 
$$\mathcal{B} = \set{\set{\phi(x_1), \ldots, \phi(x_k)} ~|~ \set{x_1, \ldots, x_k} \in \mathcal{B} }$$
\end{definition}

\begin{theorem}
Množina $D = \set{d_1, \ldots, d_k}$ je $(v, k, \lambda)$-diferenčná množina práve vtedy, keď
$(X, \mathcal{B})$, kde $X = \mathbb{Z}_v$ a $\mathcal{B} = \set{D + i ~|~ \forall i \in \mathbb{Z}_v}$ ($D + i := \set{d_1 + i, \ldots, d_k + i}$)
je cyklický $(v, k, \lambda)$-BIBD. 
\end{theorem}

\begin{definition}
\label{def:fpp1}
Nech $F$ je konečné pole. Nech $V \cong F^{n+1}$ je vektorový priestor dimenzie $n+1$ nad poľom $F$. 
Definujeme reláciu $\sim$ nad prvkami $V^* := V - \set{\vec{0}}$:

$$\forall \vec{a},\vec{b} \in V^*: \left( \vec{a} \sim \vec{b} \overset{def}{\Longleftrightarrow} \exists k \in F: \vec{a} = k \vec{b} \right)$$

Potom rozklad $V^*$ na triedy ekvivalencie $\mathbb{P}^n(V) := \faktor{V^*}{\sim}$ je $n$-rozmerná projektívna rovina nad $F$.

Projektívnu rovinu dimenzie $n$ nad konečným poľom s $q = p^r$ prvkami oznáčujeme ako $PG(n, q) := \mathbb{P}^n\left( \mathbb{Z}_p^r \right)$

\end{definition}

\begin{theorem_hard}{(Typ S dif. množín --- Singerove dif. množiny)\footnote{\TODO je to bez dokazu ci s dokazom?}}\\
Nech množina $D$ obsahuje všetky nadroviny konečnej projektívnej roviny $PG(n, q)$ 
(nadrovina je faktorový obraz vektorového podriestoru dimenzie $n$). 
Potom $D$ je $(v, k, \lambda)$-diferenčná množina s parametrami:
$$v = \dfrac{q^{n+1}-1}{q-1}, k = \dfrac{q^n - 1}{q-1}, \lambda = \dfrac{q^{n-1}-1}{q-1}$$
\end{theorem_hard}

\begin{theorem_hard}{(Typ Q dif. množín --- kvadratické reziduá, angl. \emph{Paley-type})}\\
Nech $F := GF(p^l)$ je konečné pole mohutnosti $p^l$, kde $p^l \equiv 3 \mod 4$. 
Nech $r \in F$ je generátor grupy $F^\ast := (F-\set{0}, \ast)$. 
Potom množina kvadratických reziduí grupy $F^*$ $QR(F^*) := \set{r^a \mod p^l ~|~ a \in \set{0, \ldots, p^l-1} \wedge a~\text{je párne} }$
je $(v, k, \lambda)$-diferenčnou množinou s parametrami:

$$v = p^l = 4t-1, k = 2t - 1, \lambda = t-1$$

\end{theorem_hard}

\begin{remark}

Existujú aj ďalšie triedy differenčných množín, napríklad bikvadratické reziduá alebo tzv. \emph{twin prime power difference set}.

\end{remark}

\TODO rozpísať bikvadratické rezidua, resp. twin prime power.

\section{Hadamardove matice}

\begin{definition}
Matica $H \in \set{-1, +1}^{n \times n}$ je Hadamardovou maticou rádu $n$, ak $HH^T = nI_n$ (t.j. všetky riadky sú navzájom ortogonálne).
\end{definition}

\begin{theorem}
Nech matica $H$ je Hadamardova matica rádu $n$. Potom platí:
\begin{enumerate}
    \item vymenou riadkov (stlpcov) matice $H$ dostaneme Hadamardovu maticu
    \item matica $H$ je normálna, t.j. $HH^T = H^T H$
\end{enumerate}
\end{theorem}

\begin{definition}
Hadamardova matica je v normálnom tvare, ak prvý riadok aj prvý stlpec obsahujú iba hodnoty $+1$.
\end{definition}

\begin{theorem_hard}

Nech $H$ je Hadamardova matica rádu $n$. Potom $\det{H}~=~\sqrt{n^n}$.

\end{theorem_hard}

\begin{theorem_hard}{(Hadamardov odhad)}\\
Nech $M \in \mathbb{C}^{n\times n}$ je komplexná matica typu $n\times n$, kde $\ssize{(M)_{ij}} \leq 1$. 
Nech $H$ je ľubovoľná Hadamardova matica rádu $n$.
Potom platí:

$$\det{M} \leq \det{H} = \sqrt{n^n}$$

\end{theorem_hard}

\begin{theorem}
Ak $H$ je Hadamardova matica rádu $n$, tak $n$ je buď $1$, $2$ alebo násobok $4$. 
\end{theorem}

\begin{hypothesis}{(Hadamard)}\\
$\forall n \in \set{1, 2} \cup \set{4k ~|~ k \in \mathbb{N}} \Longrightarrow$~existuje Hadamardova matica rádu $n$. 
\end{hypothesis}


\begin{theorem}{(Hadamard, Sylvester)}\\
Ak $H$, $H'$ su Hadamardove matice, tak aj $H \otimes H'$ je tiež Hadamardova matica ($\otimes$~je Kroneckerov súčin matíc).
\end{theorem}

\begin{theorem}
Normalizovaná Hadamardova matica rádu $4\mu$ existuje práve vtedy, keď existuje $(4\mu-1, 2\mu-1, \mu-1)$-diferenčná množina (typ Q). 
\end{theorem}

\section{Konečné projektívne roviny}

Jedna (algebraická) definícia konečnej projektívnej roviny (angl. \emph{finite projective plane}, alebo skrátene FPP) 
už bola daná v sekcii o diferenčných množinách (definícia \ref{def:fpp1}). V tejto sekcii uvedieme iné dve definície: 
axiomatickú a kombinatorickú.


\begin{definition}{(Axiómy konečnej projektívnej roviny)}\\
\label{def:fpp2}
Pojmy bodu a priamky sú brané ako primitívne pojmy. 
Relácie ''bod leží na priamke'' (značíme $p \in l$) a ''priamka prechádza bodom'' považujeme za primitívne relácie.

Usporiadana trojiica $\pi = (X, \mathfrak{P}, \in)$, kde $X$ je konečná množina bodov, $\mathfrak{P}$ je konečná množina priamok
a $\in$ je relácia ''patrí'' medzi bodmi a priamkami, je konečnou projektívnou rovinou, ak spĺňa nasledujúce axiómy:

\begin{enumerate}
    \item Každými dvomi rôznymi bodmi prechádza \textbf{práve 1} priamka.
    \item Každé dve rôzne priamky majú \textbf{práve 1} spoločný bod.
    \item existujú 4 body vo všeobecnej geometrickej polohe, t.j. žiadnou trojicou
    z~týchto bodov nevedie žiadna priamka.
\end{enumerate}
\end{definition}


\begin{theorem_hard}{(Desarguesova veta)}\\
\TODO obrázok
\end{theorem_hard}


\begin{theorem}
\label{th:fpp_ax_4}
V konečnej projektívnej rovine (v zmysle definície \ref{def:fpp2}) existujú 4 priamky také,
že žiadna trojica z týchto priamok nemá spoločný bod.
\end{theorem}

Čítateľ si môže všimnúť, že ak vymeníme v danom axiomatickom systéme pojmy ''priamka'' a ''bod'',
tak dostaneme ekvivalentný systém axióm. Je ľahko nahliadnuť, 
že ak v ľubovolnom platnom tvrdení o konečných projektívnych rovinách vymeníme tieto pojmy, tak
znovu dostaneme platné tvrdenie. Takéto trvdenia voláme duálne (napríklad, prvá axióma je duálna ku druhej a tretia axióma
je duálna ku vete \ref{th:fpp_ax_4}).

\begin{theorem}
Nech $\pi$ je konečná projektívna rovina 
a nech $n$ je prirodzené 
číslo väčšie alebo rovné 2. 
Potom nasledujúce tvrdenia sú ekvivalentné:
\begin{enumerate}
    \item každá priamka obsahuje práve $n+1$ bodov
    \item každý bod leží na práve $n+1$ priamkach (duálne ku 1.)
    \item nejaká priamka obsahuje práve $n+1$ bodov
    \item nejaký bod leží na práve $n+1$ priamkach (duálne ku 3.)
    \item konečná projektívna rovina $\pi$ má práve $n^2+n+1$ priamok
    \item konečná projektívna rovina $\pi$ má práve $n^2+n+1$ bodov (duálne ku 5.)
\end{enumerate}
\end{theorem}


\begin{definition}{(Kombinatorická definícia FPP)}\\
Konečná projektívna rovina rádu $n$ je $BIBD(v, k, \lambda)$ s parametrami:
$$v = n^2 + n + 1, k = n + 1, \lambda = 1$$
\end{definition}

\begin{theorem}
Kombinatorická a axiomatická definície konečnej projektívnej roviny sú ekvivalentné. 
\end{theorem}

\begin{theorem_hard}
Ak $n$ je mocninou prvočísla, tak existuje konečná projektívna rovina rádu $n$.
\end{theorem_hard}

\begin{hypothesis}
Ak existuje konečná projektívna rovina rádu $n$, tak $n$ je mocninou prvočísla.
\end{hypothesis}

\begin{definition}

Matica $C = (c_{ij})$ typu $n \times m$, kde $n \geq 4, m \geq 4$ a $c_{ij} \in \set{1, \ldots, n}$,  
má latinskú vlastnosť, ak lubovoľná podmatica z dvoch stlpcov matice $C$ nemá rovnaké riadky. Formálne,

$$\forall (i, j) \neq (k, l): (c_{ij}, c_{il}) \neq (c_{jk}, c_{jl})$$

Navyše, ak podmatica matice $C$, tvorená prvými dvomi stlpcami, obsahuje postupne všetky dvojice čísel $\set{1, \ldots, n}$
v lexikografickom poradí, tak ju voláme matica s latinskou vlastnosťou v normálnom tvare.
\end{definition}

\begin{theorem}
Nech $n \geq 3, t \geq 2$. Potom množina $t$ navzájom ortogonálnych latinských štvorcov rádu $n$ existuje práve vtedy,
keď existuje matica typu $n^2 \times (t+2)$ s latinskou vlastnosťou v normálnom tvare.
\end{theorem}

\begin{theorem}
Existencia konečnej projektívnej roviny rádu $n$ je ekvivalentná s existenciou $(n-1)$ navzájom ortogonálnych latinských štvorcov rádu $n$.
\end{theorem}

\begin{construction}
\TODO
\end{construction}

\section{Steinerovské systémy trojíc, zovšeobecnenia}

\begin{definition}
Blokové plány typu $BIBD(v, 3, 1)$ sa volajú Steinerovské systémy trojíc (angl. \emph{Steiner triplet system}, skrátene STS).
\end{definition}

\begin{remark}
Existencia STS je ekvivalentná s existenciou rozkladu kompletného grafu $K_v$ na trojuholníky.
\end{remark}

\begin{theorem}
\label{th:sts_nec}
Ak $v$ je počet objektov STS, tak $v \equiv 1 \mod 6$ alebo $v \equiv 3 \mod 6$.
\end{theorem}

\begin{theorem_hard}{(Kirkman)}\\
Ak $v$ spĺňa podmienky z vety \ref{th:sts_nec}, tak existuje STS s práve $v$ objektmi.
\end{theorem_hard}

\begin{theorem}{(Projektívne STS)\footnote{\TODO treba dôkaz?}\footnote{\TODO trochu mi nesedí: má to byť VP alebo pole (resp. zároveň)}}\\
Nech $X := \left(\mathbb{Z}_2\right)^{n+1} - \set{\vec{0}}$ je množina vektorov vektorového priestoru dimenzie $n+1$ nad poľom $\mathbb{Z}_2$ bez nulového vektora a 
$\mathcal{B} := \set{\set{\vec{x}, \vec{y}, \vec{z}} ~|~ \vec{x} + \vec{y} + \vec{z} = \vec{0}}$.
Potom dvojica $(X, \mathcal{B})$ je STS. Alternatívne, množina priamok projektívnej roviny $PG(2, n)$ tvorí STS.
\end{theorem}

\begin{theorem}{(Afinné STS)\footnote{\TODO treba dôkaz?}}\\
Nech $X := \left(\mathbb{Z}_3\right)$ je množina vektorov vektorového priestoru dimenzie $n$ nad poľom $\mathbb{Z}_3$. 
Nech $\mathcal{B} := \set{\set{\vec{x}, \vec{y}, \vec{z}} ~|~ \vec{x} + \vec{y} + \vec{z} = \vec{0}}$. Potom 
dvojica $(X, \mathcal{B})$ je STS.
\end{theorem}

\begin{theorem}{(Karteziansky súčin STS)}\\
Nech dvojice $(X, \mathcal{B})$ a $(Y, \mathcal{C})$ sú STS. Potom dvojica $(X \times Y, \mathcal{D})$, kde:
\begin{enumerate}
    \item $y \in Y, \set{b_1, b_2, b_3} \in \mathcal{B} \Longrightarrow \set{(b_1, y), (b_2, y), (b_3, y)} \in \mathcal{D}$
    \item $x \in X, \set{c_1, c_2, c_3} \in \mathcal{C} \Longrightarrow \set{(x, c_1), (x, c_2), (x, c_3)} \in \mathcal{D}$
    \item $\set{b_1, b_2, b_3} \in \mathcal{B}, \set{c_1, c_2, c_3} \in \mathcal{C}, \phi \in S_3 \Longrightarrow 
            \set{(b_1, c_{\phi(1)}), (b_2, c_{\phi(2)}), (b_3, c_{\phi(3)})} \in \mathcal{D}$ (kde $\phi$ je permutácia veľkosti $3$) 
\end{enumerate}

Potom $(X \times Y, \mathcal{D})$ je STS.
\end{theorem}


\begin{theorem}{(Vzťah STS a grupoidov)}\\
Nech $(X, \mathcal{B})$ je STS. Potom množina $X$ s binárnou operáciou $\ast$, definovanou nasledovne:
\begin{align*}
    \forall \set{x,y,z} \in \mathcal{B}:\\
    x\ast y = y \ast x = z\\ 
    x \ast z = z \ast x = y\\ 
    y\ast z = z \ast y = x\\
    x \ast x = x
\end{align*}

je idempotentný komutatívny grupoid. 
\end{theorem}


\begin{theorem}{(($2v +1$)-konštrukcia STS)}\\
Nech $(X, \mathcal{B})$ je STS a $(X', \mathcal{B}')$ je jeho disjunktná izomorfná kópia (t.j. $X \cap  X' = \varnothing$). Obraz prvku $x$ v tomto izomorfizme budeme značiť $x'$. 
Nech prvok $\infty \notin X \cup X'$.
Potom dvojica $(Y, \mathcal{C})$, kde $Y := X \cup X' \cup \set{\infty}$ a $\mathcal{C} := \mathcal{B} 
\cup \set{\set{x, y', z'} ~|~ \set{x,y,z}\in \mathcal{B}} 
\cup \set{\set{x, x', \infty} ~|~ x \in X}$, je STS.    
\end{theorem}

\TODO Paschovo prepnutie

\TODO Wilson-Schreiberova konštrúkcia

\TODO Boséova konštrukcia

\TODO Skolemova konštrukcia

\TODO Cyklické STS

\TODO Symetrické $v_3$-konfigurácie (čiastočné STS)

\begin{hypothesis}
Každý bezmostový kubický graf má 6 1-faktorov takých, že každá hrana grafu leží v práve 2 z nich.
\end{hypothesis}






